%! Author = arshiya
%! Date = 3/19/22

% Preamble
\documentclass[a4paper,11pt]{article}

% Packages
\usepackage[textwidth=7in,textheight=8.7in,headsep=2cm]{geometry}
\usepackage{fontspec}
\setmainfont{Arial}
\usepackage{amsmath}
\usepackage{tabularx}
\usepackage[table]{xcolor}
\usepackage{titlesec}
\usepackage{graphicx}
\usepackage{fancyhdr}
\pagestyle{fancy}
\usepackage{bbding}
\usepackage{lipsum}
\usepackage{hyperref}
\usepackage{xepersian}
\usepackage{import}
\settextfont{B Nazanin}
\setlatintextfont{Arial}

\lhead{\includegraphics[]{Radian_logo}}
\rhead{}
\setlength\headheight{26pt}

\titlelabel{\thetitle. }
\newcommand*\tick{\item[\Checkmark]}
\definecolor{babyblueeyes}{rgb}{0.63, 0.79, 0.95}
\title{\vspace{-2cm}\textbf{
آزمایشگاه کاربردها و شبکه‌های مطمئن و بادوام اینترنت اشیاء}\\
    \begin{center}
        گزارش خلاصه مقالات مطالعه شده
    \end{center}
    \vspace{-2cm}
}
\author{}
\date{}

% Document
\begin{document}
    \maketitle
    \thispagestyle{fancy}
    \begin{table*}[h!]
        \centering
        \begin{tabularx}{\linewidth}{|c|X|}
            \hline
            {تهیه کننده} \cellcolor{babyblueeyes} & \\ \hline
            {عنوان مقاله} \cellcolor{babyblueeyes} & \\ \hline
            {کنفرانس/ ژورنال} \cellcolor{babyblueeyes} & \\ \hline
            \multicolumn{2}{|r|}{{رتبه کنفرانس / ژورنال:} \quad {سال چاپ:}\quad {تعداد ارجاع:}{۴۳} \quad {تاریخ مطالعه:} \quad {مدت زمان مطالعه:}
                }\\ \hline

        \end{tabularx}
    \end{table*}

\import{}{sections-persian/Why.tex}
\import{}{sections-persian/Terminologies.tex}
\import{}{sections-persian/Challenges.tex}



    \section{راهکار پیشنهادی (یک پاراگراف)}\label{sec:prior-major-works}

      \begin{itemize}
        \tick {این بخش باید رویکرد کلی حل مسئله و روش ارائه شده برای چالش مطرح شده را شرح دهید. نیازی به آوردن الگوریتم، جزئیات و روابط ریاضی نیست.}
         \tick {آنچه مهم است، این است که در آینده با خواندن این بخش از خلاصه، بدانیم مقاله \lr{X} چه راه حلی ارائه کرده است تا ما راهمان را از آن جدا کنیم و یک راهکار جدید و متفاوت با آن ارائه کنیم.}
    \end{itemize}

    \section{روش‌های پیشین، که در بخش آزمایش‌ها، راهکار ارائه شده با آنها مقایسه شده است}\label{sec:assumptions}
    \begin{itemize}
        \tick {\textbf{ فقط نام روش‌ها} به همراه عنوان مقاله‌ای که آن راهکار در آن معرفی شده است.}
        \tick {این بخش از آنجا اهمیت دارد که در آینده برای انتخاب روش‌های پایه (یعنی همان روش‌هایی که قرار است آن‌ها را عیناً پیاده‌سازی کنیم و روش خودمان را با آن‌ها مقایسه کنیم) به سادگی خواهیم توانست انتخابمان را انجام دهیم.}
    \end{itemize}

    \section{فرضیات}\label{sec:evaluation}

      \begin{itemize}
        \tick { در حل مسئله و یا محیط شبیه‌سازی شده از فرضیاتی استفاده شده است، آن‌ها را در این بخش می‌نویسیم.}
        \tick {برخی مقالات ممکن است مفروضاتشان را ننوشته باشند (در این صورت برای این بخش موردی ذکر نمی‌کنیم)}
    \end{itemize}

    \section{شبیه‌ساز استفاده شده }\label{sec:limitations}

  \begin{itemize}
        \tick { شبیه‌ساز استفاده شده برای گرفتن نتایج (ذکر عنوان کفایت می‌کند مگر آنکه برای اولین بار است که با آن آشنایی پیدا کرده‌ایم. در این صورت باید با کمی جستجو در ایترنت و خود مقاله، کمی اطلاعات تکمیلی راجع به آن ابزار در این قسمت بنویسیم)}
      \tick {دلیل آوردن این قسمت در آن است که برای ما مهم است که ابزار شبیه‌سازی را انتخاب کنیم که به زبان برنامه‌نویسی مورد نظر ما نوشته شده باشد و از طرف دیگر، جامعه‌ی علمی بزرگی از آن پشتیبانی کند تا در صورتی که سوالی در فرآیند شبیه‌سازی برای ما پیش آمد، بتوانیم از آن‌ها سوال بکنیم. }
   \tick{انتخاب ابزار شبیه‌سازی یکی از مهم‌ترین انتخاب‌ها در فرآیند تحقیق یک فرد آکادمیک به شمار می‌رود.}
    \end{itemize}

    \section{ایرادات نسبت به راهکار ارائه شده و پیشنهادات برای بهبود آن}\label{sec:imporvments}

      \begin{itemize}
        \tick { در این بخش، پس از آنکه مقاله به شکل کامل مطالعه شد، چنانچه از نظر شما، نویسندگان مقاله در جایی دچار اشتباه شده‌اند، ایده دارای مشکل است، فرض مهمی را در نظر نگرفته‌اند و هر مورد دیگری که از نظر شما یک ایراد تلقی می‌شود را به صورت بولت بیان کنید.}
        \tick { ممکن است برای حل مشکلاتِ ایده‌ی مقاله و یا مواردی که نویسندگان آن‌ها را در نظر نگرفته‌اند، یک راه حل (حتی خیلی ساده) به ذهن شما برسد. حتما این موارد را در این بخش یادداشت کنید.}
        \tick { این بخش یکی از مهم‌ترین بخش‌های خلاصه‌نویسی است، زیرا موجب پایه‌ریزی ایده‌ی اصلی ما می‌شود.}
      \end{itemize}

    \section{مراجع}\label{sec:ref}

    \begin{itemize}
        \tick {در حین مطالعه این مقاله، چنانچه تعاریف، آمار و ارقام، روابط مهم ریاضی (که فکر می‌کنید در آینده ممکن است بخواهید از آن استفاده کنید)، روش‌های پیشین و هر موردی که به مقاله دیگری ارجاع داده شده بود و شما می خواهید آن را در متن خلاصه بیاورید، مرجعش را در این بخش ذکر کنید.}
        \tick {این کار به ما کمک می‌کند که در آینده مجبور نباشیم در نگارش سمینار، پایان‌نامه و یا رساله به سراغ عمل طاقت‌فرسای یافتن مراجع برویم.}
        \tick {روش ارجاع در متن گزارش‌ها به این صورت است\cite{keshav2007read}.}
        \tick {روش نگارش ارجاعات در متون علمی (مبتنی بر استاندارد \lr{IEEE}) در یک فایل جداگانه برای دانشجویان ارسال می‌شود و در یکی از جلسات هفتگی راجع به آن صحبت خواهیم کرد.}
    \end{itemize}

    \setLTRbibitems
    \bibliography{main}
    \bibliographystyle{plain}
    \resetlatinfont

\end{document}